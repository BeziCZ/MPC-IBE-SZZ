\section{Režim omezení odpovědnosti poskytovatelů služeb informační společnosti typu mere conduit, caching a hosting}

{}Relevantní právní předpisy:
\\\href{https://eur-lex.europa.eu/legal-content/CS/ALL/?uri=CELEX:32000L0031
}{Směrnice 2000/31/ES o~elektronickém obchodu}
\\\href{https://www.zakonyprolidi.cz/cs/2004-480}{Zákon 480/2004 Sb. o~některých službách informační společnosti}

Poskytovatel služeb informační společnosti (ISP) se může nacházet v~jednom ze~tří režimů odpovědnosti v~závislosti na~službách které nabízí: prostý přenos, ukládání v~mezipaměti, shromažďování informací.

Tyto principy vychází především z~faktu, že ISP nemohou za~obsah odpovídat plně (protože by to pro~ně nebylo ekonomicky udržitelné na~pojištění) ani vůbec (protože by stát neměl možnost své zákony vymáhat).

Kdy služba spoluzodpovídá je stanoveno v~autorském zákoně, v~občanském zákoníku a~dalších relevantních zdrojích; Směrnice upravuje kdy poskytovatelé odpovědní nejsou.


\subsection{Mere Conduit (Prostý přenos)}

Především poskytovatele připojení (ISP, veřejná nebo firemní Wi-Fi).
Poskytovatel \emph{není} odpovědný, pokud
\begin{enumerate}[label=\alph*)]
\item není původcem přenosu,
\item nevolí příjemce přenášené informace a
\item nevolí a~nezmění obsah přenášené informace (data jen přenáší).
\end{enumerate}

Jednotlivé členské státy mohou přikázat konkrétní komunikaci přerušit (v~ČR jde například o~nelicencované hazardní hry online), ale nesmí přikázat prohlížet obsah posílaných dat a tím kontrolovat aktivitu.

\subsection{Caching, mirroring (Ukládání v~mezipaměti)}

Jedná se především o~CDN nebo globálně dostupné služby (cloud storage).
Poskytovatel \emph{není} odpovědný, pokud
\begin{enumerate}[label=\alph*)]
\item informaci nezměnil,
\item vyhovuje podmínkám přístupu k~informaci,
\item dodržuje pravidla o~aktualizaci informace,
\item nepřekročí povolené používání technologie obecně uznávané a~používané v~průmyslu s~cílem získat údaje o~užívání a
\item ihned přijme opatření vedoucí k~odstranění jím uložené informace.
\end{enumerate}

Data smazaná na~hlavním serveru musí být odstraněna i~z~cachí a mezipamětí.

\subsection{Hosting (Shromažďování informací)}

Hosting je služba pracující s~daty generovanými uživateli (Facebook, Twittter, diskusní fórum, veřejný cloud, \dots).

Poskytovatel \emph{je} odpovědný, pokud se seznámí s~protiprávním jednáním a~přesto nekoná: jde o~princip \emph{notice--takedown}.
Musí konat okamžitě (\emph{expediciously}) jakmile se o~problematickém stavu dozví%
\footnote{
	V~Německu je \emph{okamžitě} chápáno z~hlediska poskytovatele (např.~odpovědná osoba na~dovolené se do~času nepočítá).
	U~nás se takový případ ještě k~soudu nedostal.
}%
.

V~Itálii a~Španělsku musí protiprávní skutečnost oznámit stát, ve~většině členských států to může udělat kdokoliv.
V~USA to musí být ten jehož práva jsou dotčena.

% TODO Zde je možné zmínit některé příklady:
% Napster ("Ekonomický model by bez autorskoprávně chráněného obsahu nefungoval")
% Delfi AS v. Estonia ("Posuzuje se doba od protiprávního jednání, ne od jeho nahlášení, protože provozovatel musel o urážlivých komentářích vědět.")
% eBay, Uber & Airbnb a jejich umístění na škále zprostředkovatel--poskytovatel


\clearpage
\section{Aktivní povinnosti poskytovatelů služeb informační společnosti (monitoring, filtrování)}

{}Relevantní právní předpisy:
\\\href{https://eur-lex.europa.eu/legal-content/CS/ALL/?uri=CELEX:32000L0031
}{Směrnice 2000/31/ES o~elektronickém obchodu}
\\\href{https://www.zakonyprolidi.cz/cs/2004-480}{Zákon 480/2004 Sb. o~některých službách informační společnosti}


Není možné požadovat aktivní dohled poskytovatele nad daty, ani přikázat vyhledávat skutečnosti a okolnosti poukazující na~protiprávní činnost.
Tento zákaz se ale netýká zvláštního dohledu (případ Rolex vs. ricardo.de, dohled nad~jedním uživatelem, \dots).

Obecný dohled, který směrnice/zákon zakazují nařídit, probíhá:
\begin{itemize}
    \item na~všech přenášených či ukládaných datech,
    \item vůči všem uživatelům bez rozdílu,
    \item preventivně,
    \item výlučně na náklady poskytovatele,
    \item bez~časového omezení.
\end{itemize}
Jedná se o zásah do~práv ISP (právo na~svobodu podnikání) a do~práv třetích stran (svoboda projevu a ochrana soukromí).
Aplikuje se ale prevenční povinnost a princip \emph{lex specialis}.

ISP nejsou (dle čl. 15, odst.~1 Směrnice) povinni:
\begin{itemize}
    \item dohlížet na obsah jimi přenášených nebo ukládaných informaci, ani
    \item aktivně vyhledávat skutečnosti a okolnosti poukazující na protiprávní obsah informace
\end{itemize}

Není tedy možné nakázat jakýkoliv dohled, včetně zvláštního; ISP sám ale dohled (ať už automatizovaný nebo manuální) realizovat může.

Protože se odpovědnost řídí principem \emph{notice--takedown}, pokud je ISP upozorněn na~protiprávní obsah, musí ho řešit.

Prevenční povinnost byla dokázána v~případě L'Oréal v. EBay (kde byly prodávány L'Oréal padělky): obchod byl donucen uskutečnit opatření k~prevenci dalšího porušování.


\clearpage
\section{Pojem osobního údaje, titul ke zpracování osobních údajů, zvláštní kategorie osobních údajů}

{}Relevantní právní předpisy:
\\\href{https://eur-lex.europa.eu/legal-content/CS/ALL/?uri=CELEX:32016R0679
}{Nařízení 2016/679 (GDPR) o~ochraně fyzických osob v~souvislosti se zpracováním osobních údajů a o~volném pohybu těchto údajů [\dots]}
\\\href{https://www.zakonyprolidi.cz/cs/2019-110}{Zákon 110/2019 Sb. o~zpracování osobních údajů}

Osobními údaji jsou veškeré informace o~identifikované/identifikovatelné fyzické osobě, kterou lze přímo či~nepřímo identifikovat.

Jde například o~jméno, identifikační číslo, lokační údaje, síťový identifikátor nebo na jeden či více zvláštních prvků fyzické, fyziologické, genetické, psychické, ekonomické, kulturní nebo společenské identity této fyzické osoby.

\subsection{Zpracování osobních údajů}

Článek 6 Nařízení vymezuje zákonné tituly -- podmínky, ze~kterých alespoň jedna musí být platná, aby se jednalo o~zpracovnání dle~GDPR:

\begin{enumerate}[label=\alph*)]
\item subjekt udělil souhlas se~zpracováním,
\item zpracování je nezbytné pro~plnění smlouvy,
\item zpracování je nezbytné pro~plnění právní povinnosti správce,
\item zpracování je nezbytné pro~ochranu životně důležitých zájmů subjektu,
\item zpracování je nezbytné pro~splnění úkolu ve~veřejném zájmu nebo výkonu veřejné moci,
\item zpracování je nezbytné pro~účely oprávněných zájmů správce.
\end{enumerate}

Zpracování musí být přiměřené, relevantní a~omezené na~nezbytný rozsah ve~vztahu k~účelu pro~který jsou zpracovávány (tzv. \emph{minimalizace údajů}).
Údaje být shromažďovány pro~určité, výslovně vyjádřené legitimní účely a~nesmí být zpracovávány způsobem, který je s~těmito účely neslučitelný.
Musí být zpracovávány korektně, zákonně a~transparentně.
Nesmí být zpracovávány po~dobu delší než nezbytnou pro~účely, pro~které jsou zpracovávány.
Musí být adekvátně zabezpečena.

Dle~důvodu 18 Nařízení se nevztahuje zpracování pro~osobní potřebu.
Pod~tento důvod spadají například čísla kontaktů v~telefonu nebo podpisy na~výsledcích tvorby umělecké činnosti%
\footnote{V~době zavádění GDPR se šířily poplašné zprávy, že není možné zveřejňovat autory výkresů v~mateřských školách.}.

\subsection{Zvláštní kategorie osobních údajů}

Jde o~údaje o~zdravotním stavu (očkování, zdravotní historie).
Souhlas musí být výslovný a explicitní, nestačí mít zákonný důvod.

Zvláštní kategorie osobních údajů by dle důvodu~53 Nařízení měly být zpracovávány pouze

\begin{enumerate}[label=\alph*)]
\item pro~zdravotní účely, je-li jich třeba k~dosáhnutí prospěchu fyzických osob nebo společnosti jako celku,
\item pro~účely monitorování a~varování nebo pro~účely archivace ve~veřejném zájmu,
\item pro~účely vědeckého či~historickéhov výzkumu,
\item pro~statistické účely na~základě práva Unie nebo členského státu,
\item pro~studie prováděné ve~veřejném zájmu v~oblasti veřejného zdraví.
\end{enumerate}

Zakazuje se zpracování osobních údajů, které vypovídají o~rasovém či etnickém původu, politických názorech, náboženském vyznání či filozofickém přesvědčení [\dots], a~zpracování genetických údajů, biometrických údajů za~účelem jedinečné identifikace fyzické osoby a~údajů o~zdravotním stavu či o~sexuálním životě nebo sexuální orientaci fyzické osoby.

Tento zákaz neplatí, pokud se uplatní jeden z~následujících případů (viz článek 9):

\begin{enumerate}[label=\alph*)]
\item subjekt údajů udělil výslovný souhlas,
\item zpracování je nezbytné pro~účely plnění povinnosti a~výkon zvláštních práv správce [\dots] v~oblasti pracovního práva nebo práva v~oblasti sociálního zabezpečení [\dots],
\item zpracování je nutné pro~ochranu životně důležitých zájmů subjektu [\dots],
\item zpracování provádí [\dots] nadace, sdružení [\dots] pro~vnitřní účely [\dots],
\item zpracování se týká osobních údajů zjevně zveřejněných subjektem,
\item zpracování je nezbytné pro~určení, výkon nebo obhajobu právních nároků [\dots],
\item zpracování je nezbytné z~důvodu významného veřejného zájmu [\dots],
\item zpracování je nezbytné pro~účely preventivního nebo pracovního lékařství [\dots],
\item zpracování je nezbytné z důvodů veřejného zájmu v oblasti veřejného zdraví [\dots],
\item zpracování je nezbytné pro účely archivace ve veřejném zájmu, pro účely vědeckého či historického výzkumu nebo pro statistické účely [\dots].
\end{enumerate}

\clearpage
\section{Právní postavení správce a zpracovatele osobních údajů}

{}Relevantní právní předpisy:
\\\href{https://eur-lex.europa.eu/legal-content/CS/ALL/?uri=CELEX:32016R0679#d1e3011-1-1
}{Nařízení 2016/679 (GDPR), kapitola IV: Správce a zpracovatel}

\subsection{Správce}

Správce je osoba (fyzická, právnická, orgán veřejné moci), která určuje účel zpracování osobních údajů.
Více správců může spravovat jedna data; každý z~nich zodpovídá sám za~sebe.

Musí definovat jaké údaje sbírá, na~základě čeho (titul), proč (účel) a~jak (proces).
Všechny tyto informace musí být řádně zdokumentovány.
Mohou být zpracovávány pouze osobní údaje nezbytně nutné pro konkrétní účel zpracování.
Tyto údaje nesmí být standardně bez zásahu člověka zpřístupněny neomezenému počtu fyzických osob.

V~okamžiku získání osobních údajů správce (dle článku 13 Nařízení) poskytne:

\begin{itemize}
    \item totožnost a~kontaktní údaje správce (a~jeho zástupce), případně kontaktní údaje pověřence pro~ochranu osobních údajů,
    \item účely zpracování a~jejich právní základ,
    \item příjemce nebo kategorie příjemců osobních údajů,
    \item úmysl předat osobní údaje do~třetí země nebo mezinárodní organizaci,
    \item dobu, po~kterou budou osobní údaje uloženy, případně kritéria pro~stanovení takové doby,
    \item existenci práva požadovat přístup k~osobním údajům týkajícím se subjektu údajů,
    \item existenci práva odvolat souhlas, podat stížnost u~dozorového úřadu,
    \item jestli jde o~zákonný nebo smluvní požadavek, zda má subjekt možnost údaje neposkytnout a~důsledky neposkytnutí.
\end{itemize}

Správce je odpovědný za~dodržení zásad zpracování, dodržení povinností upravených nařízením a~za~zabezpečení údajů.
Má povinnost aplikovat standardní ochrany osobních údajů, hlásit případy porušení zabezpečení údajů příslušnému úřadu a postiženým osobám nebo vést záznamy o~svých činnostech.


\subsection{Zpracovatel}

Zpracovatel je subjekt, který je pověřen správcem na~zpracování osobních údajů.
Nemusí existovat, nebo jich může být více.
Pokud začne rozhodovat o~účelu dat sám, stává se správcem.
Mezi těmito entitami musí být vždy sepsána písemná smlouva ve~které je stanoven předmět a doba trvání zpracování, povaha a~účel zpracování, typ a~kategorie údajů, práva a~povinnosti.

Zpracovatel je povinný zpracovávat údaje pouze na~základě pokynů správce, zajistit mlčenlivost osob zpracovávající údaje, zabezpečovat údaje stejně jako správce, při~rozhodnutí správce (nebo při~ukončení služeb) údaje smaže či vrátí správci.

Existují výjimky při~kterých je možné osobní údaje zpracovávat bez souhlasu subjektu:
\begin{itemize}
    \item plnění smlouvy, právní povinnosti (uchování faktury),
    \item výkon veřejné moci,
    \item ochrana~životně důležitých zájmů subjektu údajů nebo jiné FO (lékař uschovává informace o~léčbě),
    \item plnění úkolu prováděného ve~veřejném zájmu,
    \item plnění nezbytné pro~účely oprávněných zájmů příslušného správce (půjčování peněz).
\end{itemize}


\clearpage
\section{Práva subjektů osobních údajů}

{}Relevantní právní předpisy:
\\\href{https://eur-lex.europa.eu/legal-content/CS/ALL/?uri=CELEX:32016R0679#d1e2150-1-1
}{Nařízení 2016/679 (GDPR), kapitola III: Subjekt}

Subjektem údajů je člověk.


\paragraph{Informování}

Subjekt má právo být informován o~sběru osobních údajů (čl.~13, 14), na~přístup k~údajům (čl.~15) a na~opravu nepřesných a na~doplnění neúplných osobních údajů (čl.~16).


\paragraph{Výmaz a~omezení, přenos}

Subjekt má právo na~výmaz (\enquote{právo být zapomenut}) dle čl.~17, pokud

\begin{enumerate}[label=\alph*)]
\item jeho údaje již nejsou potřebné pro~deklarované účely,
\item subjekt odvolává souhlas,
\item subjekt vznese námitku proti~zpracování,
\item údaje subjektu jsou zpracovávány protiprávně.
\end{enumerate}

Toto právo se neuplatní, pokud je zpracování nezbytené pro~výkon práva na~svobodu projevu a~informace, pro~splnění právní povinnosti nebo pro~splnění úkolu provedeného ve~veřejném zájmu nebo při~výkonu veřejné moci nebo pro~určení, výkon nebo obhajobu právních nároků.

Dle čl.~18 má právo na~omezení zpracování, pokud

\begin{enumerate}[label=\alph*)]
\item subjekt popírá přesnost osobních údajů (to na~dobu potřebnou k~ověření přesnosti správcem),
\item zpracování je~protiprávní a~subjekt odmítá výmaz osobních údajů,
\item nebo správce údaje nepotřebuje, ale subjekt je požaduje pro výkon právních nároků.
\end{enumerate}

Dle čl.~20 má právo na~přenositelnost údajů jinému správci údajů.

\paragraph{Marketing, profilování a~automatické zpracování}

Dle článku 21 má právo vznést námitku proti~zpracování včetně profilování.
Subjekt má vždy právo vznést námitku proti~zpracování pro~účely marketingu.

Subjekt má právo nebýt předmětem rozhodnutí založeného výhradně na~automatizovaném zpracování.
Toto právo se nepoužije pokud to není nezbytné k~uzavření nebo plnění smlouvy, pokud je to povoleno právem Unie/členského státu nebo pokud je založeno na~výslovném souhlasu subjektu.


\clearpage
\section{Povinné subjekty dle zákona o~kybernetické bezpečnosti}

{}Relevantní právní předpisy:
\\\href{https://eur-lex.europa.eu/legal-content/CS/TXT/?uri=uriserv:OJ.L_.2016.194.01.0001.01.CES
}{Směrnice 2016/1148 (NIS) o~opatřeních k~zajištění [\dots] bezpečnosti sítí a~informačních systémů}
\\\href{https://www.zakonyprolidi.cz/cs/2014-181}{Zákon 181/2014 Sb. o~kybernetické bezpečnosti}

Kategorie povinných subjektů dle Zákona jsou:

\begin{itemize}
\item kritická informační infrastruktura: systémy na~kterých je závislá informační infrastruktura,
\item významné informační systémy: systémy spravované orgánem veřejné moci které by při~výpadku způsobily významnější problémy (interní vládní systémy: e-mail, spisová služba, zahraniční spolupráce, veřejné zakázky),
\item významné sítě: sítě se~zahraniční kontektivitou, sítě propojující kritickou infrastrukturu,
\item základní služby: jejich poskytování je závislé na~internetu a výpadek by mohl mít významný společenský nebo ekonomický dopad: energetika, doprava, bankovnictví, finanční trh, zdravotnictví, \dots
\item digitální služby: digitální tržiště, cloudové služby, vyhledávače.
\end{itemize}

Orgány a osoby, kterým se ukládají povinnosti v~oblasti kybernetické bezpečnosti, jsou dle Zákona:

\begin{enumerate}[label=\alph*)]
\item poskytovatel služby elektronických komunikací a subjekt zajišťující síť elektronických komunikací,
\item orgán nebo osoba zajišťující významnou síť,
\item správce a provozovatel informačního systému kritické informační infrastruktury,
\item správce a provozovatel komunikačního systému kritické informační infrastruktury,
\item správce a provozovatel významného informačního systému,
\item správce a provozovatel informačního systému základní služby,
\item provozovatel základní služby,
\item poskytovatel digitální služby.
\end{enumerate}

a orgán veřejné moci využívající služeb poskytovatelů cloud computingu.

Od~roku 2024 bude platit Směrnice NIS2, která některé kategorie povinných subjektů spojuje.


\subsection{Základní povinnosti subjektů}
\begin{itemize}
    \item Elektronické komunikace: kontaktní údaje, protiopatření
    \item Významné sítě: kontaktní údaje, incidenty (národní CERT), protiopatření
    \item Významné systémy: bezpečnostní opatření, incidenty (vládní CERT), protiopatření
    \item Kritická infrastruktura: bezpečnostní opatření, incidenty (vládní CERT), protiopatření
    \item Základní služby: bezpečnostní opatření, incidenty (vládní CERT), protiopatření
    \item Digitální služby: nespecifikovaná bezpečnostní opatření, incidenty s~významným dopadem (národní CERT + významný dopad na~základní službu -- vládní CERT)
\end{itemize}

\clearpage
\section{Bezpečnostní opatření, varování, reaktivní opatření a ochranná opatření dle zákona o~kybernetické bezpečnosti}

{}Relevantní právní předpisy:
\\\href{https://www.zakonyprolidi.cz/cs/2014-181}{Zákon 181/2014 Sb. o~kybernetické bezpečnosti}

Jde o~prevenční a reaktivní nástroje NÚKIBu.
Kybernetická bezpečnostní \emph{událost} vykazuje znaky, \emph{incident} je událost s~prokázanými důsledky.

\paragraph{Bezpečnostní opatření}

Jde o~realizaci performativních pravidel, tj. implementace standardů.
Performativní pravidla jsou ze~své definice nejistá: dopředu nelze vědět zda jde o~dostačující opatření.
Úřad může udělit sankci tehdy, pokud je to opak zákonného stavu (tj. nedodržení jedné z povinností stanovených samotným subjektem).
Jde například o~vnitřní normy, proškolení zaměstnanců nebo řízení vztahů s~dodavateli.

\paragraph{Varování}

Nezakládá žádnou povinnost, jenom upozorňuje na bezpečnostní riziko.
Když je ale vydané varovaní pro~povinné subjekty, subjekt ho ignoruju a vznikne mu škoda s~tím související, je za to odpovědný, protože porušil prevenční povinnost.

NÚKIB varování zveřejní na svých internetových stránkách a oznámí je povinným subjektům.

\paragraph{Ochranné opatření}

Je vydané na~základě zhodnocení bezpečnostního incidentu do~budoucnosti.

\paragraph{Reaktivní opatření}

Bezprostřední reakce na incident, má svého konkrétního adresáta a musí se vykonat ihned.
Opatření obecné povahy je naopak vládní předpis směřující k~nekonkrétnímu počtu subjektů.
U reaktivního opatření je povinnost subjektů oznámit provedení.

\paragraph{Opatření k~nápravě}

Jde o~nástroj správního práva ve~formě adresného rozhodnutí.
Ukládá se v~případech kdy Úřad zjistí nedostatky při~kontrole subjektu (např. nařízení šifrování, proškolení, zamykání, \dots)


\clearpage
\section{Procesní nástroje pro~zajištování elektronických důkazů}

{}Relevantní právní předpisy:
\\\href{https://www.zakonyprolidi.cz/cs/1961-141}{Zákon 141/1961 Sb., trestní řád}
\\\href{https://www.zakonyprolidi.cz/cs/2005-127}{Zákon 127/2005 Sb. o~elektronických komunikacích}

Důkaz je jakákoliv informace která vyvrací nebo potvrzuje libovolnou skutkovou okolnost.
Základními zásadami dokazování jsou:
\vspace*{-1em}\begin{itemize}
\item presumpce neviny,
\item zásada ústnosti (důkazy se soudci předávají ústně a prezenčně),
\item zásada veřejnosti,
\item zásada bezprostřednosti (soudce je s~důkazy seznámen bezprostředně),
\item zásada materiální pravdy,
\item zásada vyhledávací,
\item zásada volného hodnocení (důležitost důkazu je vždy subjektivní),
\item zásada zdrženlivosti (minimalizace škod, např. při~zajišťování důkazů při~domovní prohlídce),
\item proporcionalita.
\end{itemize}

\subsection{Zjednodušeně}

Neexistuje specifická úprava pro~práci s~elektronickými důkazy, používají se standardní nástroje procesního trestního práva.
Také se špatně přenáší do~vnímatelného zachycení a je nutná interpretace (soudci neumí interpretovat zdrojový kód nebo metadata).
Může také dojít k~tomu, že je důkaz nevyužitelný.

K~počítačovým datům se lze dostat třemi základními způsoby:
\begin{enumerate}[label=\alph*)]
    \item zajištění zařízení, nebo datových nosičů, na kterých jsou počítačová data uchovávána,
    \item získání přímého přístupu k počítačovým datům uchovaným v počítačových systémech,
    \item získání počítačových dat od poskytovatele služeb.
\end{enumerate}

\subsubsection{Zajištění zařízení nebo datových nosičů}

Jedná se o~jednu z~nejefektivnějších metod získání přístupu k~datům, protože existuje povinnost vydat věc která je důležitá pro~trestní řízení.

Danou věc musí vydat ten kdo ji momentálně drží, ne (pouze) její majitel; musí být poučen o~následcích neuposlechnutí; na~základě rozhodnutí mu může být odňata (k~tomu musí být přítomna nezávislá osoba a musí být sepsán protokol).

K~zajištění věci je také možné využít domovní prohlídku nebo oprávnění k~prohlídce jiných prostor (k~tomu musí být přítomen znalec a musí být sepsán protokol).


\subsubsection{Získání přístupu k~vzdáleným datům}

Pokud jsou informace volně dostupné na internetu, je možné je použít a pořizovat z nich důkazní materiál (je nutné postupovat dle pravidel a musí být sepsán protokol).

Pokud jsou data zabezpečena, lze k nim přistoupit pomocí přístupových údajů, které byly vydány dobrovolně v~rámci výslechu či vysvětlení nebo jiným způsobem.
Zabezpečená data jsou chápána jako písemnosti a~záznamy uchovávané v~soukromí.
Bez přístupových údajů je možné přistoupit jen po~povolení soudce a musí být sepsán protokol.

Pokud není možný odklad, lze k~datům přistoupit ihned a podat žádost zpětně.
Pokud povolení není získáno do~48 hodin, data musí být zničena.
Data z~mailů či chatů se berou jako odposlech.


\subsubsection{Získání dat od ISP}

Rozděluje se poskytovatel telekomunikačních služeb (data telekomunikačního provozu: telekomunikační tajemství, odposlech za~daných podmínek) a poskytovatel služeb informační společnosti.
Dále se rozlišuje i~charakter dat: různé typy dat vyžadují různé nástroje.

Data bez~povinnosti mlčenlivosti musí vydat kdokoliv.
Je povinné vyhovět dožádání orgánů činným v~trestním řízení (často to co uživatel zveřejnil, informace o~účtech, logy, metadata).
Zabezpečená data se berou jako záznamy uchovávané v~soukromí.

Na~dožádání můžou být data zachována po~určitou dobu.
Některá řízení mohou trvat dlouho a hrozí tedy smazání dat, na~požádání poskytovatel data musí uschovat.


\subsection{Detailněji rozebrané jednotlivé možnosti}

\subsubsection{Obecná součinnost (§8 TŘ)}

Státní orgány, FO, PO a dálší relevantní subjekty mají povinnost žádosti vyhovět.
Může se žádat pouze o~neutajované informace, utajované jedině se~soudním příkazem; subjekt má povinnost mlčenlivosti.


\subsubsection{Freezing (§7b TŘ)}

\emph{Freezing}: pokud hrozí ztráta, zničení nebo pozměnění důležitých dat, lze osobě držící data nařídit aby je uchovala v~nezměněné podobě pro~potřeby vydání vyšetřovatelům.

\emph{Blocking}: osobě držící data lze nařídit zablokování přístupu uživateli k~datům (maximálně 90~dní).

\subsubsection{Odposlech a záznam telekomunikačního provozu (§88 TŘ)}

Při~vedení trestního řízení pro určité zločiny a trestné činy může být vydán příkaz k~zajištění obsahu telekomunikačního provozu (data, e-mail, telefon).
Útvar zvláštních činností mívá v~sítích nainstalována zařízení které odposlech umožňují; je tedy realizován ve~spolupráci s~operátory.

Odposlech schvaluje uživatel nebo přikazuje soud.
Pokud nelze účelu dosáhnout jinak nebo pokud by to bylo moc komplikované, odposlech může být vydán i~když OČTŘ neví zda trestná činnost probíhá.


\subsubsection{Zajištění provozních a lokalizačních údajů (§88a odst. 1 TŘ)}

Při~vedení trestního řízení pro určité zločiny a trestné činy může být vydán příkaz k~zajištění metadat o~komunikaci.

\subsubsection{Data retention (§97 ZoEK)}

Poskytovatel je na základě \emph{data retention} povinen uchovávat provozní a lokalizační údaje po~dobu 6 měsíců od~doby uskutečnení daného komunikačního provozu (za~finanční odměnu), které jím nesmí být zneužity.
Pokud nejsou po~6 měsících vyžádány, musí být zničeny.


\subsection{Prohlídka, odnětí věci}

\subsubsection{Sledování osob a~věcí (§158d TŘ)}

Jde o~pátrací prostředek určený pro~zajištění operativních informací, ne~důkazů k~soudu.
Pokud je nutné u~soudu takto získané informace získat, musí o~nich být vypracován protokol.

O~vydání příkazu o~sledování osob a~věcí rozhoduje státní zástupce prostřednictvím povolení; pokud jde o~soukromá nebo utajovaná data, je nutné povolení soudce.
Při~zpětném souhlasu soudu lze provést neodkladný úkol (např. získání dat na~vzdálené službě prostřednictvím telefonu).

\subsubsection{Domovní prohlídka a~prohlídka jiných prostor}

Žádost k~ohledání musí obsahovat zdůvodnění relevantnosti místa k~trestnímu stíhání, včetně zdůvodnění proč takové informace nelze získat jiným způsobem.
Soudce žádost posoudí a~vydání či~zamítnutí odůvodní (nelze se odvolat, ale stát ručí za~případné škody způsobené nesprávným vydáním příkazu).

Prohlídka může být realizovaná pouze pokud:
\vspace*{-1em}\begin{itemize}
\item je~přiměřená: důkaz není třeba odvážet pokud může být na~místě prozkoumán vyšetřovatelem nebo znalcem,
\item je majitel přítomen nebo alespoň informován,
\item je přítomna nezúčastněná osoba, která kontroluje že nedochází k~porušení zákona,
\item je vypracován protokol (video, fotografie) který všichni zúčastnění podepisují.
\end{itemize}

Znalec i vyšetřovatel musí dodržet specifické postupy aby nedošlo ke~znehodnocení důkazů (zapečetění, zabelení do~pytle).

\subsubsection{Vydání a odnětí věci}

OČTŘ vyzývají k~předložení věci (při~odmítnutí může být odejmuta po~svolení státního zástupce).
Při~odejmutí musí být přítomna nezávislá osoba, musí být sepsán protokol a osoba musí dostat potvrzení o~odejmutí.

% TODO Je toto třeba? Nešlo by to jasněji?
% pokud jsou na zařízení data o kterých je povinná mlčenlivost (utajované informace, advokátní tajemství) je specifický postup - pouze věci né data - ovšem pokud je vydán např telefon tak naněm může být provedena forenzní analýza a tedy je možno se dostat k datům uloženým na zařízení

\subsubsection{Osobní prohlídka}

Pokud jde o~neopakovatelný úkon (podezřelý utíká z~místa činu), lze osobní prohlídku provést i bez~soudního příkazu (který je ale poté nutné získat).
Úkonu by měl předcházet výslech (stejně jako u~domovní prohlídky) s~žádostí o~dobrovolné vydání věci či předmětu.


\subsubsection{Ohledání věci}

Pozorování a~sbírání informací za~účelem objasnění věci je typicky prováděno při~domovních a~osobních prohlídkách.

Sledování konverzace na~obrazovce nelze na~základě povolení k~domovní prohlídce provádět, je nutné mít souhlas k~odposlechu.
Pokud je získán dopředu, je možné ho použít jako důkaz.


\clearpage
\section{Typy a znaky skutkových podstat počítačových trestných činů}

% FIXME Text je správný, ale vůbec nemluví o počítačových trestných činech

{}Relevantní právní předpisy:
\\\href{https://www.zakonyprolidi.cz/cs/2009-40}{Zákon 40/2009 Sb., testní zákoník}

Trestní zákoník obsahuje seznam činů pokládaných za~trestné, na~které jsou uvaleny sankce.
Při~spáchání trestného činu je nutné dokázat všechny znaky skutkové podstaty (tj. podmínky pro~jeho existenci):
\emph{kvalifikovaná podstata} jsou okolnosti jejichž přítomnost může zvýšit trest proti~základu (pomluva slovně vs. na~internetu),
\emph{privilegovaná podstata} jsou okolnosti jejichž přítomnost může snížit trest proti~základu (pouze jediný případ: vražda novorozeného dítěte matkou).

\subsection{Znaky}

Skutkovou podstatu trestného činu tvoří pět znaků a to subjekt, objekt, objektivní stránka, subjektivní stránka.

\subsubsection{Subjekt}

Subjekt je pachatel definovaný věkem a příčetností.

Nezletilé dítě právně způsobilé není: může tedy jít o~provinění, ale ne trestný čin.


\subsubsection{Objekt}

Objekt je společenský vztah, zájem nebo hodnota (právo, právem chráněný zájem) které jsou chráněny.

Meč ve~hře je také uznán za chráněný, ne~podle vlastnického práva, ale tím že je jeho \enquote{majitelem} považován za~cenný.


\subsubsection{Objektivní stránka}

Škodlivý následek existuje v~případě kdy se dá vyčíslit a po~prokázání vzniká nárok \emph{na~úhradu} (ušlý zisk).
Nemateriální újma (psychická, na~dobré pověsti) se vyčíslit nedá a nahrazuje se \emph{odškodněním}.

Protiprávní jednání je projev vůle, kterým subjekt porušuje příkazy nebo zákazy (právní povinnost), s~nimiž se spojují právní následky (uložení trestu, náhrada škody, zánik práva, \dots).

\emph{Kauzální nexus} je příčinná souvislost mezi protiprávním jednáním a škodlivým následkem.


\subsubsection{Subjektivní stránka}

Subjektivní stránka je mentální vztah pachatele k~činu a týká se otázky zda je možné zavinění prokázat.

\emph{Úmysl} je provedení něčeho zakázaného.
Rozlišuje se na přímý úmysl (plánovaná vražda) vs. nepřímý úmysl (pokus o~pouhé zranění s~neplánovaným úmrtím).

\emph{Nedbalost} je neprovedení něčeho přikázaného.
Vědomá nedbalost (\emph{\enquote{vím, ale nekonal jsem}}) vs. nevědomá nedbalost (\emph{\enquote{nevěděl jsem, ale měl jsem}}) vs. hrubá nedbalost (\emph{\enquote{nevěděl jsem, ale musel jsem}}).

\subsection{Typy}
\begin{itemize}
    \item Cyber dependent - pouze prostřednictvým počítače
    \item Cyber enabled - lze jak IRL, tak pomocí počítače
    \item Cyber supported - počítač použit jako nástroj další nelegální činnosti
\end{itemize}

\subsubsection{Hlavní skutkové podstaty:}
\begin{itemize}
    \item porušení tajemství dopravovaných zpráv (§182 TZ)
    \item porušení tajemství listin a jiných dokumentů uchovávaných v soukromí (§183 TZ)
    \item neoprávněný přístup k počítačovému systému a nosiči informací (§230 TZ)
    \item opatření a přechovávání přístupového zařízení a hesla k počítačovému systému a jiných takových dat (§231 TZ)
    \item poškození záznamu v počítačovém systému a na nosiči informací a zásah do vybavení počítače z nedpalosti (§232 TZ)
    \item neoprávněné opatření, padělání a pozměnění platebního prostředku (§234 TZ)
    \item výroba a držení padělatelského náčiní (§236 TZ)
\end{itemize}

\clearpage
\section{Subjektivní a objektivní odpovědnost}

\subsection{Subjektivní odpovědnost}

Vzniká v~případě prokázání porušení právní odpovědnosti.
Musí se prokázat subjekt (kdo je pachatel), objekt (poškozený zájem), subjektivní stránka (zavinění: úmysl či nedbalost) a objektivní stránka (škodlivý následek, protiprávní následek, příčinná souvislost).

\emph{Exkulpace} je vyvinění se: prokázání že subjekt vznik újmy nezavinil přestože jsou splněny dané předpoklady.


\subsection{Objektivní odpovědnost}

% TODO Jde to vysvětlit ještě líp?

Jde o~odpovědnost za~škodlivý následek (stav, událost), ne protiprávní jednání.

\emph{Liberace} je zproštění se objektivní odpovědnosti (vyšší moc: ke~škodě by došlo i~bez přičinění strany).
Pokud nelze liberaci užít, jde o~absolutní objektivní odpovědnost.

U~trestní odpovědnosti není objektivní odpovědnost možná a~vždy se dokazuje zavinění.


\subsection{Prvky právní odpovědnosti}

Subjektivní stránka subjektivní odpovědnosti.

Úmysl přímý vs. nepřímý.
Nedbalost vědomá vs. nevědomá vs. hrubá.


\subsection{Právní skutečnosti}

Okolnosti se kterými právní norma spojuje vznik, změnu nebo zánik právního vztahu.

Koherenční teorie: neexisuje přímý ani falzifikovatelný důkaz, ale existují důkazy nepřímé (otisky prstů, svědectví).
Nekoherenci může způsobit alibi.

Konsenzuální teorie: společnost se chová podle jistých nepsaných pravidel.


\subsubsection{Prokazatelné právní skutečnosti}

Korespondenční teorie pravdivosti: každou ze~skutečností lze prokázat (i když to může být snadné: uzavření smlouvy--dodání zboží--\emph{nezaplacení}).

\emph{Verifikace} je prokázání důkazem; \emph{falzifikace} je prokázání antiteze.


\subsubsection{Předpokládané právní skutečnosti}

Prokazují předpoklad, skutečnost konstruuje právní norma.

\emph{Doměnky} (otec dítěte je manžel matky v~době narození) vs. \emph{fikce} (skutečnost se nestala, ale právní norma říká že ano; např. fikce doručení).


\subsubsection{Známé právní skutečnosti}

\emph{Notoriety} (všeobecně známé), skutečnosti známé z~úřední povinnosti (obchodní rejstřík), skutečnosti známé z~rozhodovací činnosti (sazba ČNB).
