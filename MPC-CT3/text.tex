\section{Bezpečnost na~vrstvě L1 (bezpečnostní opatření, klíčování, dohled nad~sítí) a bezpečnost na~vrstvě L2 (bezpečnostní opatření, příklady útoků, MACsec).}

Na~fyzické vrstvě pracují technologie jako USB, Bluetooth, UTMS, 100Base-T, DSL, \dots

Na~spojové vrstvě pracují protokoly Ethernet, PPP; WEP/WPA\{,2,3\}; PAP/CHAP/EAP, \dots


\subsection{Bezpečnostní opatření na~L1}

Zajištění a~izolace kabelů nebo optiky, síťových prvků, ochrana fyzických rozhraní a portů:
\begin{itemize}
	\item Zamezení neautorizovaného připojení do~portů.
	\item Zamezení poškození síťového zařízení.
	\item Umístění bezpečnosti perimetru nebo prostoru (čidla, kamery, kontrola přístupu, uzamykatelné skříně).
	\item Pasivní ochrana: blokátory (pro~konektory a porty), klíčování (znemožnění připojení cizích skupin PatchCordů a LC konektorů).
\end{itemize}

Obfuskace komunikace (kódování, přidání bílého šumu).
Analogové šifrování (různé modulace).
Kvantová kryptografie (distribuce klíčů protokolem BB84).

Šifrování na~první úrovni způsobuje problémy (synchronizace, komplexita, neefektivní ošetření chyb).


\subsection{Dohled nad~L1 sítí}

AIM (\emph{Automated Infrastrucutre Management}) zajišťuje, dokumentuje a~monitoruje síťovou kabeláž.
Umožňuje detekovat vložení a odstranění propojovacích kabelů (pomocí mikrospínače nebo například měřením impedančních vlastností).

IPLMS (\emph{Intelligent Physical-Layer Management Solution}) kombinuje inteligentní propojovací panely se softwarovými funkcemi a poskytuje informace o~stavu připojení na~portech.


\subsection{Rizika a~hrozby na~L1}

Externí útoky: modifikace dat (\emph{tampering}), rušení (\emph{jamming}), odepření služeb (DoS), odposlech (\emph{sniffing}; bezdrátově, metalikou, optikou, prostorově pomocí kamery nebo mikrofonu), obejití přístupu (\emph{authentication bypass}; lámání hesel, obnovení do~továrního nastavení, nouzový režim, reboot), útok postranním kanálem (proudově, napěťově, zvukově, opticky).

Interní útoky: poškození kabelů nebo zařízení, neoprávněné připojení k~portu (\emph{tap}, \emph{splitter}) a odposlech či modifikace dat.

% TODO Spectre, Meltdown (?)


\subsection{Bezpečnostní opatření na~L2}

Základní bezpečnostní opatření linkové vrstvy:
\vspace*{-0.5em}\begin{itemize}
    \item Bezpečnost komunikace bod--bod nebo bod--více bodů v~jedné doméně (autentizace, klíčový management, šifrování, kontrola integrity a autentičnosti dat).
    \item Šifrování obsahu po fyzickou adresu (MAC).
    \item Bezpečnost přepojování rámců.
\end{itemize}

Ethernet packet obsahuje pouze CRC32, nelze ho považovat za~bezpečnostní prvek.

802.1x: kontrola přístupu zařízení do~sítě na~portech nebo na~bezdrátových přístupových bodech.
Doporučuje se EAP a integrace s~AAA (Radius, Diameter).

802.11: WEP i WPA obsahují mnoho chyb, WPA2 (802.11i): KRACK (2017), WPA3 (2019): zranitelnost v~Dragonfly handshake.

Konkrétní opatření:
\vspace*{-0.5em}\begin{itemize}
    \item autentizace 802.1x (filtrace MAC lze obejít spoofingem),
    \item hardening síťových zařízení,
    \item ochrana přístupu na~porty (Autentizace EAP),
    \item vypnutí nepoužívaných portů,
    \item stanovení portů vedoucích k~DHCP serverům,
    \item funkce \emph{sticky MAC},
    \item \emph{dynamic ARP inspection} (DAI; prevence proti ARP spoofingu).
\end{itemize}

\subsection{Rizika a~hrozby na~L2}

DoS (vyčerpání MAC adres, kolize), odposlech, modifikace dat.

% TODO VLAN útoky? jak si to představit, co to znamená?
Existuje množství útoků: spoofing MAC adres, \emph{ARP cache poisoning}, \emph{ARP cache flooding}, WEP/WPA zranitelnosti, VLAN útoky.

% TODO Příklady útoků


\subsection{MACSec}

802.1ae je specifikace implementace SecY (\emph{MAC Security Entities}).
Zajišťuje důvěrnost dat, autenticitu i~integritu.
Jako takový nezajišťuje management klíčů a ustanovení bezpečné relace; existuje 802.1x-2010 jako MKA (\emph{\mbox{MACSec} Key Agreement}).

802.1ae+802.1x dohromady zajišťují oboustrannou autentizaci, výměnu klíčů, šifrování, integritu a autentičnost.
Jsou součástí Linuxového jádra 4.5+, některých CISCO přepínačů a dalších L2 prvků.

Protože jde o~ochranu na~L2, chrání data jen uvnitř jednoho LAN segmentu a při~přechodu do~jiné sítě je terminován.
Podporuje VLAN.

Rámce jsou podobné Ethernet rámcům, ale obsahují navíc \emph{Security Tag} a MAC.
Používá se AES-GCM-128 (nebo -256).


\clearpage
\section{Bezpečnost na~vrstvě L3 (bezpečnostní opatření, příklady útoků, IPsec, bezpečnost IPv6).}

L3 je první vrstva na~které je zavádění šifrování praktické i pro~běžné použití.

Zajištění autentizace/autorizace komunikujících stran (PKI, AAA, ACL), důvěrnost (VPN), bezpečnost služeb (ICMP, ARP, IGMP), QoS, bezpečnost protokolů (OSPF, RIP).


\subsection{Bezpečnostní opatření na~L3}

Zabezpečení.
Key management (PKI), autentizace (AAA řešení, Radius, Tacacs, Kerberos), ACL, firewally.

Důvěrnost a integrita.
IPSec.
Bezpečnost je založena na~vyšších vrstvách (TLS, DTLS, QUIC).

Ochrana funkčnosti sítě.
Směrovací protokoly nemívají možnost vzájemné autentizace (OSPFv2, RIPv2, EIGRP, BGP: zcela bez autentizace, případně MD5 PSK).


\subsection{Rizika a~hrozby na~L3}

Spoofing, IP flooding, ICMP flooding, Smurf DDoS (podvržení zdroje v~ICMP dotazu), wormhole, blackhole, sybil.

U~dynamických směrovacích protokolů je možné do~sítě nasadit škodlivé zařízení které může síť ovládnout.


\subsection{IPSec}

IPSec poskytuje bezpečnost na~třetí vrstvě včetně managementu klíčů a ověření protistran.
Jde o~sadu několika protokolů: \emph{Authentication Header} (integrita), \emph{Encapsulating Security Payloads} (šifrování), \emph{Security Associations} (parametry spojení).
Lze ho využívat v~tunelovacím (vše šifrované, nová hlavička) nebo transportním (šifrovaná data) módu.
Ustanovení klíče je možné přes PSK, IKE1/IKE2, Kerberos.

Implementován ve~Windows, OSX, nových verzích Android, pro~Linux \{strong,libre,open\}swan.
Alternativou k~IPSec je např. Wireguard.

3 části:
\begin{itemize}
    \item authentication header (integrita, autentizace, integrita vlastně zajišťuje i~správnost adres)
    \item encapsulating security payload (šifruje)
    \item security association (databáze spojení)
\end{itemize}

Módy ESP:
\begin{itemize}
    \item transportní (IP packet nedotčen, v něm IPSec packet)
    \item tunel (celý IP packet zašifrován, obalen IPSec a IP packetem)
\end{itemize}


\subsection{Bezpečnost IPv6}

V~IPv6 měl být IPSec původně vyžadován, dnes jde však jen o~doporučení.
IPv6 používá nové přístupy a~protokoly které zvětšují prostor pro~možné útoky.
Některé útoky z~v4 lze adaptovat: ARP $\rightarrow$ NDP, broadcast $\rightarrow$ multicast multiplikace, fragmentace (směrovače $\rightarrow$ uzly).
Je možné zneužívat rozšířené hlavičky.

Kvůli výrazně většímu prostoru je v6 odolná vůči skenování; 64b prefix, různé způsoby přidělování adres.

% TODO Bezpečnost multicastu



\clearpage
\section{Bezpečnost TCP (útoky, protiopatření), protokol TLS -- Transport Layer Security (princip, součásti, příklady útoků).}

TCP hlavička obsahuje protichybový kontrolní součet.
Při~navazování spojení se provádí třícestný handshake:
\begin{center}
\begin{tabular}{rcl}
	klient & & server \\
	\hline
	SYN, SEQ=$X$ & $\rightarrow$ & \\
	& $\leftarrow$ & SYN, ACK=$X+1$, SEQ=$Y$ \\
	SEQ=$X+1$, ACK=$Y+1$ & $\rightarrow$ \\
\end{tabular}
\end{center}
Při~uzavírání spojení obě strany zasílají FIN, ACK.
Zamítavá odpověď je RST.


\subsection{Útoky na~TCP}

\textbf{Spoofing.}
Asociačním stavem se myslí kombinace portu a~inicializačního sekvenčního čísla, které lze predikovat nebo hádat.
Ochranou je dobrá implementace TCP stacku (a PRNG v~něm), IPSec nebo např. TCP Auth Option (RFC~5925).

\textbf{Skenování portů.}
Skenováním lze zjistit jaké služby na~stanici běží, a~fingerprinting techniky umožňují detekovat i verze systému a služeb (\texttt{nmap}, \texttt{netstat}; \href{https://github.com/hackman/shijack}{\texttt{Shijack}}, \href{https://linux.die.net/man/1/hunt}{\texttt{Hunt}}).
Přímá ochrana neexistuje, je možné skenování ztížit nasazením edge proxy která serverům a jejich službám posílá jen žádaný provoz a zbytek je terminován u~ní.

\textbf{Únos TCP relace.}
Varianta spoofingu, jde o~zastavení klienta (DoS) a odhad sekvenčních čísel jeho komunikace se~serverem.
Při úspěšném únosu TCP relace je komunikace pouze jednosměrná: od~útočníka na~server.
Úspěšný útok není jednoduché provést, protože inicializační sekvenční číslo je těžké predikovat. 

Ochranu představuje:
\vspace*{-0.5em}\begin{itemize}
    \item ochrana proti slepému ukončení spojení (RFC 5961),
    \item používání Challenge ACK paketů a omezení ACK paketů za~sekundu.
\end{itemize}

\textbf{SYN DoS.}
Zaslání mnoha SYN paketů bez následného potvrzení navázání spojení ACK zprávou.
Tím na~serveru zůstanou polootevřená spojení.
Ochranou jsou SYN cookies (hashování adres, portů a času; výsledek v~sekvenčním čísle), snížení časovače, navýšení délky fronty, recyklace nejstarších polootevřených spojení, filtrace nových spojení.


\vfill
\subsection{Protokol TLS}

\emph{Transport Layer Security}.
Jde o~mezivrstvu zajištující bezpečný přenos L7 protokolů nad~TCP.
Nabízí jedno- i dvoucestnou autentizaci s~využitím certifikačních autorit, obousměrnou symetricky šifrovanou konverzaci, kontrolu integrity a autentičnosti (HMAC, CMAC, AEAD).

Skládá se z~protokolů \emph{Record Layer Protocol} (dělení dat, komprimace, šifrování), \emph{Handshake Protocol} (autentizace stran, dohoda algoritmů a tajemství), \emph{ChangeCipherSpec} (přechod na~jinou šifrovací sadu), \emph{Alert} (signalizace).

Při~navazování se provádí čtyřcestný handshake:
\begin{center}
\begin{tabular}{rcl}
	klient & & server \\
	\hline
	ClientHello & $\rightarrow$ & \\
	& & ServerHello \\
	& & Certificate \\
	& $\leftarrow$ & ServerHelloDone \\
	ClientKeyExchange & & \\
	ChangeCipherSpec & & \\
	Finished & $\rightarrow$ & \\
	& & ChangeCipherSpec \\
	& $\leftarrow$ & Finished \\
\end{tabular}
\end{center}
Výsledná relace může obsahovat jedno nebo více TCP spojení.

\begin{itemize}
    \item ClientHello obsahuje cipher suite a~client random.
    \item ServerHello obsahuje session ID, server random
    \item Klient musí ověřit certifikát
    \item Premaster secret vytvoří klient, zašifruje veř.\,klíčem serveru (např.\,u~RSA), pro jiné suites jiné mechanismy výpočtu (třeba DH)
    \item Z premaster secret vypočtou obě strany secret key
\end{itemize}





\subsection{Útoky na~TLS}

\textbf{Padding Oracle} (2002).
Chyba v~návrhu SSL (MAC-then-encrypt v~CBC).
Server vyzrazuje jestli je padding správný.

\textbf{POODLE} (Padding Oracle on Downgraded Legacy Encryption; 2014).
Degradace na~SSL~3 a~návazný Padding Oracle.
Ochranou je využít autentizované režimy šifer (AES-GCM), použití encrypt-then-MAC.

\textbf{Renegotiation} (2009).
Nový handshake v~rámci existujícího spojení který nahradil ten původní.
Ochranou je ho zakázat.

\textbf{CRIME} (Compression Ratio Infoleak Made Easy; 2002).
Získání cookies v~HTTPS a SPDY spojení; kompresní metoda DEFLATE nahrazuje opakované bajty a sledováním velikostí odpovědí lze získat informace o~šifrovaném obsahu.
Ochranou je zakázat kompresi.

\textbf{BREACH} (Browser Reconnaissance and Exfiltration via Adaptive Compression of Hypertext; 2013).
Obdoba CRIME využívající HTTP kompresi místo TLS komprese.

\textbf{HeartBleed} (2014).
Chyba v~implementaci OpenSSL, konkrétně v~heartbeat zprávě.
Klient zasílá dotaz obsahující data a~jejich délku a server odpovídá stejnou zprávou.
Zranitelnost HeartBleed spočívala ve~faktu že server použil zadanou délku zprávy a mohl odpovědět částí dat ze~své paměti.


\clearpage
\section{Bezpečnost UDP (útoky, protiopatření, zabezpečení nad protokolem UDP), DTLS (princip, účel), bezpečnost DNS, protokol DNSSEC, šifrování zpráv DNS.}

UDP je bezstavový protokol bez~garance doručení a kontroly pořadí; hlavička obsahuje pouze porty, délku a kontrolní součet.
DTLS umožňuje data šifrovat, autentizovat i~kontrolovat.
QUIC integruje TLS~1.3 a podporuje až 0RTT komunikaci.

Záplavové a amplifikační DDoS útoky:
DNS flood, SSDP flood, NTP flood, SNMP flood, UDP fragmentation flood, VoIP flood, \dots
Ochranou je limitace UDP odpovědí a zahazování nevyžádaného UDP provozu ve~firewallu.

\textbf{Memcrashed} (2018).
\href{https://memcached.org}{Memcached} je cachovací systém urychlující načítání webových stránek (klíč--hodnota, na~způsob Redis).
Šlo o~chybu implementace umožňující až 51\,000$\times$ amplifikaci (DNS amplifikace je zhruba padesátinásobná, NTP šedesátinásobná).


\subsection{Zabezpečení nad~UDP}

UDP lze balit do~TLS (DTLS).

IETF QUIC je protokol postavený nad~UDP.
Vyžaduje použití TLS~1.3, součástí handshake je také TLS handshake, což umožňuje 2RTT až 1RTT komunikaci (protože součástí třetí a čtvrté zprávy již jsou i~data).
Umožňuje dynamicky měnit adresy na~síťové a transportní vrstvě.

\subsection{DTLS}
\begin{itemize}
    \item Stejně jako TLS: integrita, utajení, autentičnost.
    \item Sekvenční čísla -- řeší ztrátovost UDP
    \item Taky má connection ID a~délku payloadu
    \item Verze obdobné jako TLS
    \item Využito třeba v~SRTP, lze multicastovat (TLS umí jen unicast)
\end{itemize}

Jeden z~možných handshakes (podle suite)
\begin{center}
\begin{tabular}{rcl}
	klient & & server \\
	\hline
	ClientHello & $\rightarrow$ & \\
	& \(\leftarrow\) & HelloVerifyRequest (with cookie) \\
    ClientHello (with cookie) & \(\rightarrow\) & \\
    & \(\leftarrow\) & ServerHello \\
	& & ServerCertificate \\
    & & [ServerKeyExchange] \\
    & & [CertificateRequest] \\
    & & ServerHelloDone \\
	ClientKeyExchange & \(\rightarrow\) & \\
    {[}Cert{]} & & \\
    CertVerify & & \\
    ChangeCipherSpec & & \\
	Finished & & \\
	& \(\leftarrow\) & ChangeCipherSpec \\
	& & Finished \\
\end{tabular}
\end{center}

\subsection{Bezpečnost DNS}

DNS je hierarchický systém překladu doménových jmen a IP adres, používající port 53.
Existuje třináct kořenových serverů s~vysokou globální redundancí.

Ani v~současné době není ve~výchozím stavu šifrovaný kvůli důrazu na~rychlost; dochází k~únikům detailů o~provozu a ohrožení soukromí.
Jde o~prostor pro~tunelování provozu (SSH over DNS) i útoků.

Dosáhnout lepšího soukromí lze dosáhnout různými způsoby.
Omezení rekurze záleží na~rekurzivním resolveru a ne klientech: DNS dotaz obsahuje pouze data nutná k~vyřešení dotazu (root DNS $\rightarrow$ \texttt{cz.}, DNS pro~\texttt{.cz} $\rightarrow$ \texttt{vut.cz.}, DNS pro~\texttt{vut.cz} $\rightarrow$ \texttt{fekt.vut.cz.}, \dots).

\emph{Bailiwick} je jednoduchá kontrola DNS záznamů podle správnosti domén a odmítnutí dalších informací DNS, pokud server není autorizován k~odpovědi pro~dotazovanou doménu.

DNS dotazy lze balit do~TLS nebo HTTPS paketů  (DoT, DoH), což prvkům infrastruktury znemožní sledovat DNS provoz a chování koncových stanic.
Dochází tím ale k~nárůstu objemu dat a~času nutnému k~vyřízení požadavku.

\textbf{DNS spoofing} je zaslání falešné odpovědi klientovi.
\textbf{DNS Cache Poisoning} je otrava cache a potenciální útok na~více klientů najednou.

\textbf{Zone transfer} je operace synchronizace Master a Slave DNS serverů v~rámci organizace.
Pokud je přenos zóny nakonfigurován nedostatečně, útočník může o~zónu požádat a získat tím seznam existujících subdomén.


\subsubsection{DNSSec}

Ochrana proti podvržení a manipulaci s~DNS dotazy.
Klient (nebo rekurzivní server) ověřuje odpovědi (původ a integritu) pomocí digitálních podpisů; nejde o~zajištění ochrany proti odposlechu nebo odepření přístupu.
Časová razítka zajišťují ochranu proti opakování.

Veřejný klíč zóny je zapsán v~doménových informacích u~nadřazené autority, čímž dochází ke~zformování řetězce důvěry.

RRSIG je podpis odpovědi, DNSKEY je veřejný klíč použitý k~podpisu, DS slouží k~ověření u~nadřazené autority, NSEC a NSEC3 prokazují neexistenci požadované domény.


\clearpage
\section{Zabezpečení v~sítích typu Low-Power Wide Area Network (kryptografické ochrany, problémy a~omezení), zabezpečení v mobilních sítích 2G -- 5G (základní principy bezpečnosti pro~2G, 3G a 4G), bezpečnost a slabiny optických sítí.}

LP WAN je reakce na~narůstající počet IoT technologií.
Jsou vhodné pro~omezené uzly (nízká spotřeba, malá cena, dlouhý dosah).
Jejich hlavní nevýhodou je nízká úroveň zabezpečení a nízká datová propustnost.
Jejich použití je vhodné zejména pro~smart metering, pet/property tracking, bezpečnost, smart cities, \dots

Kryptograficky nejsou přenosy chráněny buď vůbec, nebo používají symetrické klíče; asymetrické algoritmy bývají pro~jejich hardware moc náročné.
Některé sítě nabízí i~rotaci klíčů a~E2EE.


\subsection{LP WAN}

Jde o~výrazně heterogenní prostor s~velkým množstvím alternativních řešení.
Zprávy mívají kontrolu integrity, ale chybějící nebo slabé šifrování.


\subsubsection{LoRa}

Rychlost 292~bps až 50~kbps.
Využívá proprietární modulaci a bezlicenční pásma s~dosahem přes 10~km.

Sítě má veřejné i~soukromé.

Specifikace LoRaWAN je otevřená.
Zařízení komunikují asynchronně; buď ze~své vůle nebo při~stanoveném intervalu.
Pakety přijaté sítí jsou filtrovány proti~duplicitám, zkontrolovány a předány aplikačním serverům.

Pro~zabezpečení používá symetrickou kryptografii; autentizace (Device ID), integrita (MIC), zabezpečení (AES-128-CCM, sekvenční číslo -- zabezpečeno i~proti opakování zpráv).

Umožňuje šifrování v~RF i~end-to-end. Klíč relace \uv{AppSKey} odvozen z~hlavního klíče zařízení \uv{AppKey}.


\subsubsection{SigFox}


Je vhodná pro~sběr dat a senzoriku.
Denní limit 140 zpráv o~velikosti 12~B a čtyři potvrzovací zprávy o~velikosti 8~B.

Pro~zabezpečení používá symetrickou kryptografii; autentizace (Device ID, MAC), integrita (MAC, sekvenční číslo), zabezpečení (není, případně AES-128-CTR).


\subsubsection{NB-IoT}

Úzkopásmová LTE komunikace.
50 kbps až 250 kbps s~nízkou latencí.
Náročnější než výše zmíněné (autentizace SIM kartou, vyšší přenosové rychlosti), vyšší cena.

Bezpečnost je založena na~3GPP a LTE protokolech uvnitř SIM.


\subsection{Mobilní sítě}

SIM obsahuje 128b (od~4G 256b) předsdílený klíč.

% TODO Tato část potřebuje výrazně rozšířit. Bylo by vhodné uvést některé z historických útoků a specifikovat konkrétní algoritmy.


\subsubsection{2G: GSM}

Šifrování prodouvou šifrou, autentizace jednostrannou \emph{challenge--response}, integrita nezajištěna.
Identifikace pomocí IMSI na~SIM, pseudonymita pomocí TMSI (dočasný síťový identifikátor%
\footnote{Při~zapnutí SIM oznámí své IMSI v~otevřeném tvaru, síť odpoví s~šifrovaným TMSI, kterým se poté zařízení identifikuje. Falešné BTS může IMSI odposlechnout a zprávy podvrhovat.}%
).

Šifrování proudovými šiframi, autentizace jednostranně, integrita nezajištěna.

\begin{itemize}
    \item Autentizace pomocí klíče na~SIM (algoritmus A3).
    \item A5: šifrování (64 b klíč)
    \item A8: generování klíče pro A5
\end{itemize}



\subsubsection{3G: UMTS}

Šifrování datové komunikace i signalizace pomocí AES (128 b), oboustranná autentizace, integrita, na~čipové kartě je logická aplikace USIM.

Používá se algoritmická sada Milenage: autentizace (f1, f2), generování klíčů (f3, f4, f5), šifrování (f8; AES-CTR+OFB), integrita (f9).


\subsubsection{4G: LTE}

4G je packetová síť (EPS: \emph{Evolved Packet System}) přímo na~TCP/IP: SIP, RTP, IPSec, SRTP, DTLS, TLS.
Využívá podobný AKA protokol jako 3G (ale s~256b klíčem), proudovou šifru Snow 3G, blokovou šifru AES.



\subsubsection{5G}

V~síti existují nejen plošné, ale i~vertikální buňky (drony, letový provoz, IoV).

Bezpečnost založena na~ozkoušených protokolech z~4G, oboustranná autentizace, kontrola integrity i autentičnosti, mitigace a prevence známých útoků.
Byly odděleny a segmentovány části (snížení mitigace útoků).
% TODO Používá Snow-V
% https://www.ericsson.com/en/blog/2020/6/encryption-in-virtualized-5g-environments


\subsection{Bezpečnost a slabiny optických sítí}

Optické sítě tvoří fyzickou vrstvu – bezpečnost na této vrstvě není prioritou, počítá se s bezpečnostní ve vyšších vrstvách ISO/OSI modelu či TCP/IP modelu. 


\subsubsection{Typické hrozby a zranitelnosti}

\begin{itemize}
    \item Odposlech dat: neoprávněný přístup k~datům a k~hlavičkám rámců (pasivní útok; např. pomocí splitteru nebo využití ohybu vlákna).
    \item Narušení služeb v~optické síti: narušení datového přenosu nebo záměrná degradace fyzických parametrů v~optických sítí (aktivní útok; útočník musí disponovat speciálním zařízením pro~generování signálu).
    \begin{itemize}
        \item in-band jamming: rušivý signál v~konkrétním pásmu
        \item out-of-band jamming: je potřeba větší výkon rušivého signálu, ruší více pásem naráz, ale ne plným výkonem)
    \end{itemize}
    \item (D)DoS \emph{((Distributed) Denial of Service)}: narušení služeb pomocí DDoS útoků může nastat u~optických sítí využívající SDN \emph{(Software Defined Network)}.
    \item Narušení sítě: útok zneužívající např. slabou autentizaci a~autorizaci uživatelů. 
\end{itemize}

Odposlechy snižují úroveň signálu a~tak lze detekovat útok.
Narušení lze poznat podle zvýšeného BER (množství chyb).

\subsubsection{Slabiny}

Inicializace v~GPON: výměna klíče mezi OLT a ONU v~plaintextu (rok 2015).

Další útoky v~GPON:
\vspace*{-0.5em}\begin{itemize}
    \item Možnost modifikace zpráv PLOAM (physical layer operation administration and maintenance)(chybí autentizace a integrita zpráv).
    \item Falešné OLT (optical link termination -- na straně poskytovatele): odposlech upstream, MitM útoky.
    \item Krádež služby (falešné ONU (optical network unit -- na straně zákazníka)): útočník s~přeprogramovaným ONU se vydává za jiné ONU.
\end{itemize}

\clearpage
\section{Ochrana~soukromí v~ICT (pojmy, typy anonymizace), kryptografické metody zajištující ochranu soukromí, síťové metody poskytující ochranu soukromí a anonymizační nástroje a systémy.}

V~kontextu ICT se ochranou myslí zejména ochrana soukromí osob proti nekontrolovanému sběru, ukládání nebo uvolňování osobních údajů.

Pojmy: anonymizace (skrytí identity), pseudonymizace (částečné skrytí identity), anonymita (stav znemožňující identifikaci), osobní údaje (jméno, pohlaví, věk, datum narození, IP adresa, fotografie), citlivé osobní údaje (orientace, rodné číslo, zdravotní stav), další osobní informace (bydliště a lokalizace, finanční situace), profilování (propojování jednotlivých informací), \emph{Privacy by Default}, \emph{Privacy by Design}.

Hlavními vlastnostmi soukromí jsou \textbf{nespojitelnost}, \textbf{průhlednost} a \textbf{možnost intervence}.
Ochrana soukromí lze dělit do osmi kategorií:
\textbf{informování} uživatele o~nakládání a zpracování jejich informací, \textbf{kontrola} nad~informacemi (odstranění, (ne)souhlas), \textbf{minimalizace} a limitace sběru a zpracování, \textbf{oddělení} informací a prevence proti korelacím, \textbf{skrývání} informací (proti nepovolaným stranám), \textbf{zabstraktnění} informací (limitování detailů), \textbf{demonstrace} opatření, \textbf{vyžadování} ochrany.

Právně je ochrana soukromí zajištěna GDPR (příp. CCPA v~USA/Kalifornii),%
\footnote{Více ve~státnicových otázkách z~MPC-ODP.}
ISO/IEC 29100.

\begin{itemize}
    \item Anonymizace klienta
    \item Anonymizace serveru
    \item Oboustranná anonymizace
\end{itemize}


\subsection{Kryptografické metody ochrany soukromí}

E2E šifrování.
Atributová autentizace, ZK schémata, sdílená tajemství.
Skupinové, kruhové, slepé podpisy.
Homomorfní funkce a operace nad~zašifrovaným obsahem.
\emph{Multiparty computation}.
K-anonymita, maskování dat, mikrodatová ochrana.


\subsection{Síťové metody ochrany soukromí}

Privátní spojení a překlad adres: TLS, DoT/DoH.
VPN, Proxy.
Tor.

Anonymní profily prohlížečů.
Rozšíření prohlížečů limitující sledování (blokování cookies, reklam, rozšíření sociálních sítí).

\subsection{Anonymizační nástroje a systémy}

Nástroje L3 + L4:
\vspace*{-0.5em}\begin{itemize}
    \item VPN aplikace (nahrazení IP adres, šifrování obsahu dat).
    \item Onion routing implementace (např. nástroj Tor)
    \item MixNets nástroje.
    \item Proxy servery.
    \item Zabezpečené přenosy souborů.
\end{itemize}

Nástroje L7:
\vspace*{-0.5em}\begin{itemize}
    \item Anonymní prohlížeče webu (anonymous browsers).
    \item Antitrakovací nástroje a rozšíření prohlížečů (blokování cookies, tlačítek sociálních sítí, reklamy).
    \item Anonymní instant messaging a chatovací nástroje.
    \item Anonymní emailové nástroje.
    \item Anonymní vyhledávače.
    \item Anonymní autentizace (Uprove, Idemix).
\end{itemize}

Systémy:
\vspace*{-0.5em}\begin{itemize}
    \item Tor \emph{(The Onion Router)}
    \item Loopix: vkládá do~zpráv drobná, nezávislá zpoždění a generuje krycí provoz
    \item Java Anon Proxy: založena na~metodě kaskády MixNetů
    \item Idemix: limituje předávané informace (atributy) na~ty které služba opravdu potřebuje k~provozu.
\end{itemize}

\clearpage
\section{Forenzní analýza (hlavní cíle, základní principy, vysvětlete časové značky a časovou osu událostí).}
\subsection{Hlavní cíle}

Forenzní analýza se zabývá zajištěním podkladů pro~analýzu a interpretaci nalezených informací a prezentací vysledků incidentu.
Hlavními cíly jsou:
\vspace*{-0.5em}\begin{itemize}
    \item poskytnout spolehlivé informace,
    \item podporovat řešení bezpečnostních incidentů,
    \item podporovat možné eskalace případů (soudy).
\end{itemize}

\subsection{Základní principy}

Je nutné nezměnit analyzovaný systém (tj. disk naklonovat a potom pracovat pouze s~kopií).
Každý krok se musí dokumentovat.
Každý nález musí být identifikován, zaznamenán, každý důležitý soubor by měl být označený hashem.

Cílem je získat plný řetězec návazností: spojení vstupních dat s~výsledkem je zdrojem důvěryhodnosti celkové analýzy.
Závěry musí být ověřitelné i~dalším nezávislým analytikem.

Dokumentace musí být neutrální, zaměřena pouze na~fakta a~jejich interpretaci, nesmí hodnotit nebo soudit.
Výsledek musí být srozumitelný i pro~ne zcela technické publikum, musí být vhodně interpretován a logicky organizován.

\begin{table}[ht]
\centering
\onehalfspacing
\begin{tabular}{ll}
definice úlohy
& specifikace vstupních dat \\
& specifikace výstupních dat (odpovědí na~otázky) \\
& poznámky zadavatele \\
\hline
provedení analýzy
& použité nástroje a~postupy \\
& dokumentace jednotlivých kroků \\
& zjištění odpovědí na~otázky \\
\hline
prezentace výsledků
& závěrečná zpráva \\
& zhodnocení \\
& prezentace výsledků \\
\end{tabular}
\caption{Struktura odevzdávané dokumentace.}
\end{table}
\FloatBarrier


\subsection{Časová značka}

Většina operací v~operačním systému je svázána s~konkrétním časem (spuštění programu, úprava nebo přístup k~souboru, \dots).
V~souborovém systému i žurnálu tak zůstává mnoho stop po~operacích: spuštění programu, úprava souboru, log běžící služby.

Každý souborový systém obsahuje vedle dat i~metadata o~nich (název, velikost, typ, časové značky modifikace a přístupu, vlastníci).


\subsection{Časová osa události}

Časová osa agreguje všechny zajímavé události (nebo jejich vybranou podmnožinu) na~jednom místě pro~účely zjištění korelací.

Protože může jít o~data z~více zdrojů (stanice, servery, síť), je nutné všechna zařízení udržovat co nejblíže časově, aby byla korelace možná.
Velmi často je také nutné vše převést na~UTC, protože zařízení mohlo být v~různých časových pásmech.


\clearpage
\section{Bezpečnostní analýza zdrojového kódu (základní princip, fáze analýzy, výhody, limity, používané nástroje).}

Jedná se o~automatizované nebo manuální bezpečnostní testování. Provádí se v~rámci DevSecOps (devops (prostředky pro rychlé vytvoření, testování a~nasazení kódu) + security). Hledá slabiny nebo škodlivý kód.

% Hlavním cílem analýzy je poskytnutí informací, které jsou potřebné kvůli \textbf{vytvoření důkazního materiálu} a \textbf{obraně proti stejnému incidentu v~budoucnu}.  

% \subsection{Základní dělení analýzy malware}

% (celá tahle část je ze staré prezentace)

% Cílem analýzy je zjistit, jak daný škodlivý kód funguje. Proto se analýza dělí na:

% \subsubsection{Statická analýza}

% Hlavním cílem je pochopit strukturu programu a jeho funkčnost bez spuštění.
% Je bezpečnější než dynamická.
% Je považována za~základní analýzu malware.
% \textbf{Skenování antivirovými programy} slouží na~určení zda je vzorek malware či nikoli, porovnáním s~databázemi.

% Autoři škodlivého kódu využívají různé techniky k~maskování:

% \begin{itemize}
%     \item Obfuskační techniky -- zmatení konstantami a podmínkami.
%     \item Šifrovací techniky - využití entropie.
% \end{itemize}

% Nástroje statické analýzy:

% \begin{itemize}
%     \item \textbf{PEiD} -- základ statické analýzy, ovládaní pomocí GUI, výsledkem je detekce .exe souborů, kompilátorů a nástrojů pro~šifrování.
%     \item \textbf{GT2} –- príkazový řádek, detekuje modifikátory a kompilátory podle binárního podpisu.
%     \item Dependency Walker, HexDrive, strings, DIE, pestudio, IDA, \dots
% \end{itemize}


% \subsubsection{Dynamická analýza}

% Dochází ke~spuštění programu v~kontrolovaném prostředí (\emph{sandbox}) a pozorování jeho chování.
% Detekují se změny provedené malwarem v~systému, je potřeba rozlišit normální změny a změny způsobené škodlivým kódem.


% Nástroje dynamické analýzy:

% \begin{itemize}
%     \item \textbf{Hook-based} -- zapíšou se do~API funkcí a zaznamenávají změny probíhající v~systému. Jsou to nástroje s~největší škálou možností.
%     Např. \textit{Process Monitor, pymon.py}.
%     \item \textbf{Difference-based} -- pořídí obraz před a po~spuštění a porovnávají změny.
%     Např. \textit{Regshot, Winanalysis}.
%     \item \textbf{Notification-based} -- zaregistrují oznámení při určité události.
%     Např. \textit{Process Monitor, Preservation}.
% \end{itemize}

% Metody detekce změn systému:

% \begin{itemize}
%     \item \textbf{Sledování API funkcí} -- zaznamenávají tok mezi programy a OS.
%     Funkce nástroje: \textit{Hook-based}.
%     \item \textbf{Logování neúspěšných akcí} -- nahlášení neúspěšných pokusů o~provádění změn.
%     Funkce nástroje: \textit{Hook-based}.
%     \item \textbf{Logování dočasných souborů} -- zaznamenání vytváření dočasných souborů.
%     Funkce nástroje: \textit{Hook-based, Notification-based}.
%     \item \textbf{Rozlišování mezi různými typy modifikací} -- zaznamenání změn souboru.
%     Nástroj \textit{Hook-based} je schopen rozpoznat, zda byla změněna velikost souboru a atributy.
%     \item \textbf{Ukázka změn v téměř reálném čase} -- nástroje \textit{Hook-based, Notification-based} ukazují změny v systému okamžitě.
%     \textit{Difference-based} vykazují změny až po vytvoření druhého otisku systému.
%     \item \textbf{Zobrazení procesů zodpovědných za provedení změn} -- pouze \textit{Hook-based} mohou identifikovat procesy (název, ID) odpovědné za~provedení změny.
%     \item \textbf{Zobrazení prováděných změn v časové ose} -- nástroje \textit{Hook-based, Notification-based} zaznamenávají aktivity v~pořadí ve~kterém k~nim dochází.
% \end{itemize}

% Příklady nástrojů:

% \begin{itemize}
%     \item
%     	\textbf{Process Monitor PM} zachycuje Windows API funkce.
%     	Je to hybrid mezi Hook-based a Notification-based.
%     	Umožňuje sledovat všechny soubory, registry a procesy na~Windows.
%     \item
%     	\textbf{Regshot} patří mezi Difference-based nástroje.
%     	Detekuje změny registrů.
%     \item
%     	\textbf{Debugger x64dbg} při spouštění a krokování vzorku dochází k~jeho reálnému spuštění.
% \end{itemize}

\subsection{Fáze DevSecOps}
\begin{itemize}
    \item SDLC = secure development lifecycle:
    \item commit -- psát bezpečný kód
    \item pull request
    \item code review -- tady přichází analýza člověkem
    \item CI/CD -- tady analýza nástroji
    \item QA, integrační testy -- tady dynamická analýza (penetrační testování)
    \item deployment, delivery -- taky jde dělat pentesty
    \item opakuje se pro další změny kódu
\end{itemize}

\subsection{Statická analýza}
\begin{itemize}
    \item Nespouští kód
    \item Fáze:
    \begin{itemize}
        \item kód
        \item Tvorba modelu (lexikální analýza, syntaktická analýza)
        \item statická analýza (pravidla, vyžadují sémantickou analýzu, symbolickou exekuci, abstraktní simulaci, analýzu toku dat a~graf toku řízení)
        \item výsledky
    \end{itemize}
\end{itemize}

Například: Nemá nedůvěryhodný zdroj možnost předávat parametry kritické funkci? (SQL query bez sanitizace?)

\subsubsection{Výhody}
\begin{itemize}
    \item Nedostatky ještě před buildem
    \item CI/CD i lokálně
    \item Whitebox, rychlé odhalení možných slabin
    \item Výstupy jde použít v reportu
\end{itemize}
\subsubsection{Limity}
\begin{itemize}
    \item Nutnost porozumět výstupům
    \item Nejde bez zdrojového kódu
    \item Všechny nálezy nutné otestovat (odfiltrovat false-positives)
    \item Nástroje nemusí podporovat všechny jazyky nebo funkce
    \item Různé konfigurace můžou způsobit různé výsledky
    \item Neúplný model nebo default settings tvoří falešně pozitivní výsledky
    \item Limitace nástrojů (neodhalí vše)
\end{itemize}
\subsubsection{Nástroje}
\begin{itemize}
    \item Snyk
    \item Bandit
    \item GitLab SAST
    \item SonarQube
    \item Fortify
\end{itemize}



\clearpage
\section{Penetrační testování (rozdíly v penetračním testování realizovaným metodou Ad-hoc a pomocí metodologie, hlavní cíle ASVS metodologie, bezpečnostní úrovně) a testování bezpečnosti webové aplikace (zranitelnost Path traversal).}

Penetrační testování je provedení testu s~cílem identifikovat zranitelnosti, které by mohly být přítomny v~počítači, serveru, v~informačním systému, síti, aplikaci nebo v~organizaci.
Penetrační testy odhalují zranitelnosti i~hrozby, jakými by mohla být aktiva zneužita a poskytují návod, jak snížit existující riziko.

Stupnice závažností zranitelností: informační, nízká, střední, vysoká, kritická.

Black box -- grey box -- white box.


\subsection{Testování metodou Ad-hoc}

Jedná se o~nahodilé testování, kde tester využije své znalosti k~otestování aplikace.
Není přesně definován průběh a rozsah testu, zadavatel testu neví, co vše bude testováno.
Zadavatel takzvaně kupuje zajíce v pytli (řídit se může pouze referencemi a kvalifikací testera).


\subsection{ASVS metodologie}

\textit{Application Security Verification Standard} OWASP definuje úrovně testování (level 1 -- 3), které jasně formulují oblasti, které musí být během testu prověřeny.
Zadavatel i~tester tak ví co bude obsahem testování.

ASVS má dva cíle:
\begin{itemize}
    \item pomáhat organizacím vyvíjet a udržovat bezpečné aplikace,
    \item umožnit prodejcům bezpečnostních služeb a zadavatelům jasně definovat požadavky a nabídky.
\end{itemize}

3 úrovně ověřování zabezpečení:
\begin{itemize}
    \item Úroveň 1 -- pro nízké úrovně zabezpečení, plně testovatelná penetrací, obstojí proti OWASP top 10.
    \item Úroveň 2 -- pro aplikace, které obsahují citlivá data, která vyžadují ochranu. Útočníci nejspíše budou zruční
    \item Úroveň 3 -- pro nejkritičtější aplikace, které provádí transakce, obsahují citlivá lékařská, vojenská data atd.
\end{itemize}


\subsection{Testování webových aplikací}

Manuální testování je zdlouhavé a drahé, ale nezbytné pro~zjištění co nejvíce informací.
Využívá znalostí provádějící osoby a hledají se bezpečnostní zranitelnosti.

K~pokrytí běžných nedostatků se používá automatické testování, které nejde tolik do~hloubky.

Ideálně se využívá obou metod, napřed automatizované testování pro~zjištění známých a~naskriptovaných nedostatků systému.
Poté se využije znalosti testera v~manuálním testováním.

\subsubsection{Automatizace}

Automatizační nástroje jsou rychlé a levné, ale mají nízkou pravděpodobnost nalezení opravdových nedostatků, mohou vyvolávat falešný pocit bezpečí.
Kvůli svým technickým omezením většinou nemohou nalézt
zero-days zranitelnosti,
chyby v~logice,
nedostatečnou autorizaci,
chyby které se projevují na~jiných místech nebo
chyby které lze vyvolat pouze složitým testovacím scénářem.

\subsubsection{Path traversal}

Zranitelnost, která umožňuje útočníkovi pohybovat se v~adresářích souborového systému serveru, na~kterém běží aplikace.

Pomocí sekvence znaků \texttt{../} se lze dostat až do~kořenového adresáře a přistupovat tak k~souborům na~disku.


\clearpage
\section{Standardy v~kyberbezpečnosti (ISO, PCI DSS, FIPS, známé organizace), certifikace a hodnocení kyberbezpečnosti (Common Criteria, OWASP).}

Kybernetickou bezpečnost je možné rozdělit na:
\vspace*{-1em}\begin{itemize}
\item softwarovou bezpečnost,
\item počítačovou bezpečnost (OS, data),
\item síťovou bezpečnost (rozhraní, protokoly),
\item fyzickou (přístupové a dohledové systémy),
\item bezpečnost procesů a řízení bezpečnosti informací.
\end{itemize}
Hodnocení bezpečnosti by mělo korespondovat s~předpisy a~normami.

Standartizace a~normy v~oblasti bezpečnosti ustanovují hlavně
ISO (\emph{International Organizatio for Standartization}),
IEC (\emph{International Electrotechnical Comission}) a~národní organizace (NIST, BSI, \dots).


\subsection{ISO/IEC 27001}

Norma ISMS (\emph{Information Security Management Systems}) je součást rodiny ISO/IEC 27k.
Definuje požadavky na~systém managementu bezpečnosti informací, především řízení bezpečnosti důvěry informací pro~zaměstnance, procesy, IT systémy a firemní strategie.
Je kompatibilní s~dalšími normami a certifikacemi (např. GDPR, \dots).

Norma rozlišuje pět skupin informačních aktiv: informace, hardware, software, prostory, lidé.


\subsection{PCI DSS}

Bezpečnostní standard PCI DSS (\emph{Payment Card Industry Data Security Standard}) představuje mezinárodní pravidla definující podmínky nakládání s~údaji držitelů platebních karet.
Cílem je omezit rizika úniků dat a~snížit jejich zneužití.

PCI DSS je modelový rámec pro~zajištění bezpečnosti který určuje nejvhodnější postupy pro~minimalizaci rizika odcizení dat.
Bývá vyžadován kartovými asociacemi a~společnostmi které zpracovávají, přenášejí nebo uchovávají data držitelů platebních karet.

PCI DSS specifikuje dvanáct požadavků pro~splnění, které jsou organizovány do~šesti základních skupin:
\vspace*{-0.5em}\begin{enumerate}
\item \textcolor{red}{Vytvořit a udržovat bezpečnou síť a systémy.}
\item \textcolor{orange}{Chránit data uživatele karty.}
\item \textcolor{brown}{Udržovat program managementu zranitelností.}
\item \textcolor{olive}{Implementovat silné metody kontroly přístupu.}
\item \textcolor{teal}{Pravidelně monitorovat a testovat sítě.}
\item \textcolor{blue}{Udržovat bezpečnostní politiku informačních systémů.}
\end{enumerate}

Požadavky na PCI DSS jsou:
\vspace*{-0.5em}\begin{enumerate}
\item \textcolor{red}{Instalace a údržba konfigurace firewallu pro ochranu dat uživatelů karet (tj. uživatelská data).}
\item \textcolor{red}{Nepoužívat defaultní hesla a jiné parametry nastavené prodejcem/dodavatelem komponent.}
\item \textcolor{orange}{Ochrana uložených dat uživatelů karet.}
\item \textcolor{orange}{Šifrování přenosu uživatelských dat přes otevřené spoje a veřejné sítě.}
\item \textcolor{brown}{Celková ochrana všech systémů proti malware včetně pravidelných updatů AV programů.}
\item \textcolor{brown}{Vývoj a údržba bezpečnostních systémů a aplikací.}
\item \textcolor{olive}{Omezení přístupu k uživatelským datům na potřebné minimum nutného k provedení služby.}
\item \textcolor{olive}{Identifikace a autentizace přístupu k systémovým komponentům.}
\item \textcolor{olive}{Omezení fyzického přístupu k uživatelským datům.}
\item \textcolor{teal}{Sledování a monitorování všech přístupů k síťovým zdrojům a uživatelským datům.}
\item \textcolor{teal}{Pravidelné bezpečnostní testování systémů a procesů}
\item \textcolor{blue}{Zajištění pravidel adresující informační bezpečnost pro fyzické osoby.}
\end{enumerate}


\subsection{Metodiky hodnocení bezpečnosti zařízení a~systémů}

\subsubsection{OWASP}

\emph{Open Web Application Security Project} je celosvětová nezisková organizace zaměřená na~softwarovou bezpečnost.
Veškeré jimi vydávané dokumenty a~software jsou volně dostupné.

Dokumentační projekty:
\vspace*{-0.5em}\begin{itemize}
    \item OWASP Application Security Verification Standard: testování a bezpečný vývoj webových aplikací.
    \item OWASP AppSensor: framework a~metodologie pro~implementaci detekce průniku.
    \item OWASP Software Assurance Maturity Model (SAMM): open framework pro~organizace pro~zajištění software security.
    \item OWASP Top Ten: awareness dokument pro~bezpečnost webových aplikací.
    \item OWASP Testing Guide: \emph{best practice} penetration testing framework.
\end{itemize}


\subsubsection{Common Criteria}

Standard ISO/IEC 15408 pro~certifikaci a hodnocení produktů počítačové bezpečnosti.
CC dávají jistotu že se proces specifikace, implementace a hodnocení bude řídit přísným a standardizovaným způsobem.

Dává jistotu, že proces specifikace, implementace a hodnocení produktu počítačové bezpečnosti se bude řídit přísným a standardizovaným způsobem:
\vspace*{-0.5em}\begin{itemize}
    \item Požadavky na~specifikaci mohou snášet samotní uživatelé (komunity), bezpečnostní atributy a implementaci zajišťují výrobci produktů.
    \item Testovací laboratoře musí vyhovět ISO 17025 (kontrolováno národní schvalovací autoritou, např. BSI nebo NIST).
\end{itemize}

Součásti tohoto procesu jsou:
\vspace*{-0.5em}\begin{itemize}
\item
	Cíl hodnocení (TOE, \emph{Target of Evaluation}):
	produkt nebo systém, který je předmětem hodnocení.
\item
	Ochranný profil (PP, \emph{Protection Profile}):
	identifikuje bezpečnostní požadavky pro~určitou třídu zařízení a který je příslušný danému uživateli ke~konkrétnímu účelu.
\item
	Bezpečnostní cíl (ST, \emph{Security Target}):
	identifikuje bezpečnostní vlastnosti cíle hodncení.
	Může se vztahovat na~jeden nebo více ochranných profilů.
\item
	Úroveň bezpečnosti (EAL, \emph{Evaluation Assurance Level}):
	specifikuje výslednou úroveň.
	Každá úroveň EAL odpovídá balíčku SAR, který pokrývá kompletní vývoj produktu s~danou úrovní přísnosti.
\end{itemize}


\subsubsection{FIPS}

FIPS 140-2 je standard který definuje bezpečnostní požadavky na~kryptografické moduly a~jejich software a~hardware komponenty.

Definuje čtyři úrovně bezpečnosti (\emph{Security Levels}):
\begin{enumerate}
\item
    Základní bezpečnostní požadavky pro~moduly.
    Bez specifikace fyzické bezpečnosti.
\item
	Specifikuje také požadavky na~fyzickou bezpečnost modulu (např. ochrany proti manipulaci, \emph{tamper-evident} ochrany, zabezpečení úložiště klíčů atp.)
\item
	Specifikuje také požadavky na~bezpečnost kritických bezpečnostních parametrů v~modulu.
	Zajištění smazaní parametrů při~detekci průniků atd.
\item
	Nejvyšší úroveň; kompletní fyzická bezpečnost proti aktivním útočníkům.
\end{enumerate}


\subsubsection{Známé organizace}
Známé národní organizace pro bezpečnost:
\begin{itemize}
    \item NIST (National Institute of Standards and Technology) - USA
    \item American National Standards Institute (ANSI) - USA
    \item British Standards Institution (BSI) - UK
    \item German Federal Office for Information Security (BSI) – DE
    \item Federal Information Processing Standards (FIPS) – USA
\end{itemize}


\subsection{Legislativa}

Zákon č.~181/2014 Sb., o~kybernetické bezpečnosti stanovuje subjekty kterých se týká (poskytovatelé elektronických komunikačních služeb, významné sítě, kritická infrastruktura) a určuje povinnosti, vymezuje pojmy, určuje úlohy národních organizací (NBÚ, NÚKIB).

Vyhláška č.~316/2014 Sb., o~kybernetické bezpečnosti definuje pojmy uvedené obecněji v~ZoKB.
Obsahuje výčet konkrétních opatření které jsou dotčené organizace povinny učinit aby splnily jednotlivé body zákona.

Vyhláška č.~317/2014 Sb., o~významných informačních systémech konkrétně stanovuje na~koho se kybernetický zákon vztahuje.


\subsubsection{Legislativa ve~světě}

NIST (\emph{National Institute of Standards and Technology}, USA),
ANSI (\emph{American National Standards Institute}, USA),
BSI (\emph{British Standards Institution}, UK),
BSI (\emph{Bundesamt für Sicherheit in der Informationstechnik}, DE),
FIPS (\emph{Federal Information Processing Standards}, USA),
ENISA (\emph{European Union Agency for Cybersecurity}, EU),
PCI DSS (\emph{Payment Card Industry Data Security Standards}.
